\documentclass[a4paper,12pt]{article}
\usepackage[utf8]{inputenc}
\usepackage[T2A]{fontenc}
\usepackage[russian, english]{babel}
\usepackage{graphicx}
\usepackage{listings}
\usepackage{xcolor}
\usepackage{hyperref}

\lstset{
    basicstyle=\ttfamily\small,
    breaklines=true,
    inputencoding=utf8,
    extendedchars=true,
    literate={а}{{\cyra}}1 {б}{{\cyrb}}1 % Добавьте другие символы, если нужно
}

\title{Руководство пользователя Taskman}
\author{Марченко Т.Д.}
\date{\today}

\begin{document}

\maketitle
\tableofcontents

\section{Введение}
Taskman — простое консольное приложение для управления задачами, написанное на Bash. Основные возможности:
\begin{itemize}
\item Добавление задач с описанием
\item Пометка задач как выполненных
\item Просмотр списка задач с нумерацией
\item Удаление задач по номеру
\item Поддержка установки через Homebrew
\end{itemize}

\section{Установка и запуск}
\subsection{Требования к системе}
\begin{itemize}
\item Операционная система: macOS, Linux или WSL
\item Bash версии 4.0+
\item Homebrew (для установки через пакетный менеджер)
\end{itemize}

\subsection{Установка через Homebrew}
\begin{enumerate}
\item Добавить пользовательский репозиторий:
\begin{lstlisting}[language=bash]
brew tap teamofeyy/tap
\end{lstlisting}

\item Установить приложение:
\begin{lstlisting}[language=bash]
brew install taskman
\end{lstlisting}
\end{enumerate}

\begin{figure}[h]
  \centering
  \includegraphics[width=0.5\textwidth]{images/install.png}
  \caption{Пример установки через homebrew}
  \label{fig:install}
  \end{figure}

\subsection{Ручная установка}
\begin{enumerate}
\item Скачать репозиторий:
\begin{lstlisting}[language=bash]
git clone https://github.com/Teamofeyy/Taskman.git
\end{lstlisting}

\item Установить права выполнения:
\begin{lstlisting}[language=bash]
chmod +x Taskman/bin/taskman
\end{lstlisting}
\end{enumerate}

\subsection{Создание задач}
\begin{lstlisting}[language=bash]
taskman add create documentation
\end{lstlisting}

\subsection{Просмотр задач}
\begin{lstlisting}[language=bash]
taskman list
\end{lstlisting}
Пример вывода:
\begin{verbatim}
1. [ ] Создать документацию
2. [x] Написать базовый функционал
\end{verbatim}

\subsection{Пометка задачи выполненной}
\begin{lstlisting}[language=bash]
taskman update 1
\end{lstlisting}

\subsection{Удаление задачи}
\begin{lstlisting}[language=bash]
taskman remove 2
\end{lstlisting}


\begin{figure}[h]
  \centering
  \includegraphics[width=0.5\textwidth]{images/usage.png}
  \caption{Пример работы с taskman}
  \label{fig:usage}
  \end{figure}

\section{Пример реализации}
\begin{lstlisting}[language=bash, caption=Фрагмент скрипта taskman]
case $1 in
  add)
    echo "[ ] ${@:2}" >> "$TASKS_FILE"
    echo "Added: ${@:2}" ;;
  list)
    nl -w2 -s'. ' "$TASKS_FILE" ;;
  update)
    sed -i '' "${2}s/\[ \]/[x]/" "$TASKS_FILE" ;;
esac
\end{lstlisting}

\section{Справочник команд}
\begin{table}[h]
\centering
\begin{tabular}{|l|l|l|}
\hline
\textbf{Команда} & \textbf{Описание} & \textbf{Пример} \\
\hline
add & Добавить задачу & taskman add "Тест" \\
list & Показать задачи & taskman list \\
update & Пометка выполнения & taskman update 1 \\
remove & Удаление задачи & taskman remove 2 \\
\hline
\end{tabular}
\caption{Таблица команд}
\end{table}

\section{Решение проблем}
\subsection{Ошибка "No such file or directory"}
Убедитесь что файл:
\begin{lstlisting}[language=bash]
ls -l $(which taskman)
\end{lstlisting}

\subsection{Задачи не сохраняются}
Проверьте права на файл:
\begin{lstlisting}[language=bash]
chmod 666 ~/.taskman_tasks
\end{lstlisting}

\subsection{Обновление версии}
\begin{lstlisting}[language=bash]
brew upgrade taskman
\end{lstlisting}

\end{document}

